\message{ !name(一道不等式题目2014_5_26.tex)}\documentclass[a4paper]{article}
\usepackage{amsmath,amsfonts,amsthm,amssymb}
\usepackage{bm}
\usepackage{hyperref}
\usepackage{geometry}
\usepackage{yhmath}
\usepackage{pstricks-add}
\usepackage{framed,mdframed}
\usepackage{graphicx,color} 
\usepackage{mathrsfs,xcolor} 
\usepackage[all]{xy}
\usepackage{fancybox} 
\usepackage{xeCJK}
\newtheorem{theo}{定理}
\newtheorem*{exer}{不等式}
\newenvironment{theorem}
{\bigskip\begin{mdframed}\begin{theo}}
    {\end{theo}\end{mdframed}\bigskip}
\newenvironment{exercise}
{\bigskip\begin{mdframed}\begin{exer}}
    {\end{exer}\end{mdframed}\bigskip}
\geometry{left=2.5cm,right=2.5cm,top=2.5cm,bottom=2.5cm}
\setCJKmainfont[BoldFont=FZHei-B01S]{FZFangSong-Z02S}
\renewcommand{\today}{\number\year 年 \number\month 月 \number\day 日}
\begin{document}

\message{ !name(一道不等式题目2014_5_26.tex) !offset(-3) }

\title{\huge{\bf{一道IMO不等式预选题的新证明}}} \author{\small{叶卢
    庆\footnote{叶卢庆(1992---),男,杭州师范大学理学院数学与应用数学专业
      本科在读,E-mail:yeluqingmathematics@gmail.com}}}
\maketitle
\begin{exercise}[第31届IMO预选题,泰国提供]
  已知 $a,b,c,d\in \mathbf{R}^{+}$,且 $ab+bc+cd+da=1$,求证
\begin{equation}\label{eq:1}
\frac{a^3}{b+c+d}+\frac{b^3}{a+c+d}+\frac{c^3}{a+b+d}+\frac{d^3}{a+b+c}\geq \frac{1}{3}.
\end{equation}
\end{exercise}
\begin{proof}[\textbf{证明}]
$$
ab+bc+cd+da=1 \iff (a+c)(b+d)=1.
$$
不妨先让 $b+d$ 固定成任意一个正实数,即让 $b+d=p$,其中 $p\in \mathbf{R}^{+}$ 是一个常数.则
$a+c$ 也固定,$a+c=q=\frac{1}{p}$.则
\begin{align*}
  \frac{a^3}{b+c+d}+\frac{c^3}{a+b+d}&=\frac{a^3}{p+c}+\frac{c^3}{p+a}\\&=\frac{(a^4+c^{4})+p(a^3+c^3)}{p^2+1+ac}.
\end{align*}
由于
$$
ac\leq \frac{(a+c)^2}{4},
$$
$$
a^3+c^3=(a+c)^3-3ac(a+c)\geq (a+c)^3-\frac{3(a+c)^3}{4},
$$
$$
a^4+c^4=(a+c)(a^3+c^3)-ac(a+c)^2-2a^{2}c^2\geq
(a+c)\left[(a+c)^3-\frac{3(a+c)^3}{4}\right]- \frac{(a+c)^4}{4}-2 \left[\frac{(a+c)^2}{4}\right]^{2}.
$$
上面的三个不等式的等号成立的条件都是 $a=c$.因此我们有
$$
  \frac{a^3}{b+c+d}+\frac{c^3}{a+b+d}=\frac{(a^4+c^{4})+p(a^3+c^3)}{p^2+1+ac}
$$
在 $a=c=\frac{q}{2}$ 的时候达到最小值,最小值为
$$
\frac{q^4}{4+2q^{2}}.
$$
同理,
$$
\frac{b^3}{a+c+d}+\frac{d^3}{a+b+c}
$$
在 $b=d=\frac{p}{2}$ 的时候达到最小值.最小值为
$$
\frac{p^4}{4+2p^{2}}.
$$
为了证明不等式 \eqref{eq:1},我们只用证明
\begin{equation}\label{eq:2}
\frac{q^{4}}{4+2q^{2}}+\frac{p^4}{4+2p^2}\geq \frac{1}{3}.
\end{equation}
其中 $pq=1$.将不等式 \eqref{eq:2} 左边化简,
\begin{align*}
  \frac{q^{4}}{4+2q^{2}}+\frac{p^4}{4+2p^2}&=
\end{align*}
\end{proof}
\end{document}








\frac{4(a^4+b^4)+\frac{1}{2}(a^2+b^2)}{2a^2+2b^2+\frac{5}{4}}&=\frac{4(a^2+b^2)^{2}-\frac{1}{2}+\frac{1}{2}(a^2+b^2)}{2(a^2+b^2)+\frac{5}{4}}\\&=\frac{4(a^2+b^2)-\frac{1}{2(a^2+b^2)}+\frac{1}{2}}{2+\frac{5}{4(a^2+b^2)}}













且等号成立时,$ab=\frac{1}{4}$.而
\begin{align*}
\frac{2a^3}{a+2b}+\frac{2b^3}{2a+b}&=\frac{2a^3(2a+b)+2b^3(a+2b)}{(a+2b)(2a+b)}\\&=\frac{4(a^4+b^4)+2ab(a^2+b^2)}{2a^2+2b^2+5ab}\\&=\frac{4[(a^2+b^2)^2-2a^{2}b^{2}]+2ab(a^2+b^2)}{2(a^2+b^2)+5ab}\\&=\frac{4[(a^2+b^2)-\frac{2a^{2}b^{2}}{a^2+b^{2}}]+2ab}{2+5
\frac{ab}{a^2+b^2}}\\&\geq \frac{4ab}{3}.
\end{align*}
做到这里,不得不承认失败了.
\message{ !name(一道不等式题目2014_5_26.tex) !offset(-113) }
